\documentclass{homework}

\title{Your Title}
\author{Y.\,O. Urname}

\begin{document}

\maketitle

\exercise
The exercises are automatically numbered, starting from one. Packages such as \texttt{amsmath} and \texttt{hyperref} are included by default.

Paragraphs are not indented, but are instead separated by some vertical space.

As an example: the \emph{standard inner product} on $\R^n$ is defined as
\[\vec a * \vec b \coloneqq a_1 b_1 + \dots + a_n b_n\qquad\text{for }\vec a,\vec b \in \R^n.\]
Note that \texttt{*} can be used instead of \verb|\cdot|, and \verb|\R| instead of \verb|\mathbb{R}|. (For a normal asterisk, use \verb|\ast|.) Of course, there are also macros for the natural numbers etc. Commands such as \verb|\abs{}| and \verb|\set{}| can be used to create (scaled) delimiters. For example,
\[\abs{\frac{1}{1 - \lambda h}} \le 1\qquad\text{and}\qquad\set{x \in \R \mid 1 < \sqrt{x^3 + 2} < \frac{3}{2}}.\]
The starred version of these commands disables the auto-scaling.

\exercise*
Each exercise (except the first) starts on a new page. You can disable this behavior using the starred version of the command.

\exercise[10]
Optionally, you can specify the number of points for an exercise.

For more information, refer to \url{https://github.com/gijs-pennings/latex-homework}.

\end{document}
